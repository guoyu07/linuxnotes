\part{Alpine Linux}


\chapter{Overview}

Alpine Linux是一个基于musl和BusyBox的Linux发行版,其最早是一个LEAF project的一个fork,二者的分歧在于LEAF期望产生一个更小体积的发布版,而Alpine希望在减小体积的同时仍然能够支持重量级的软件包(例如Squid和Samba等),并且能够支持更新的Linux内核。

Alpine Linux在默认的Linux内核上打了PaX和grsecurity补丁。



Alpine Linux compiles all user space binaries as position-independent executables with stack-smashing protection.

作为可选的支持,ACF(Alpine Configuration Framework)是一个类似于Debian debconf的应用程序配置框架,可以用来配置一个Alpine Linux Machine。

Apline Linux使用的C标准库已经从最初的uClibc转换成了musl,它可以二进制方面部分实现对glibc的兼容。




\section{Package Managerment}


Alpine Linux使用自己的包管理器——apk-tools,并且包含了大多数通用的软件包(例如GNOME、Xfce和FireFox等)。

\section{Run-from-RAM}

Alpine Linux可以被定制安装为从内存启动,而且LBU(Alpine Local Backup)工具可以将所有配置文件备份到一个APK overlay(通常缩写为apkovl)的tar.gz文件,默认情况下将所有已更改文件的副本存储在/etc中。

\section{Init System}

Alpine Linux使用OpenRC而非systemd作为初始化工具。


\section{Container}

默认的Alpine Linux(除内核外)只有4~5MB,因此部署Alpine的容器大小可以缩减到8MB左右。

Alpine Linux比大多数分发基础镜像都小得多,不过Alpine Linux使用的musl libc可能导致某些软件遇到问题(取决于libc要求的深度),不过大多数软件没有这个问题。

基于Alpine的最小化镜像往往并不包含git或bash等工具,因此需要在自己的Dockerfile中添加需要的工具。


\section{Security}

默认的Alpine Linux核心中包含了PaX和grsec,可以减少类似于vmsplice()本地root漏洞的攻击的影响,而且所有的Alpine Linux软件包都使用了堆栈溢出保护编译来减轻用户态缓冲区溢出的影响。





