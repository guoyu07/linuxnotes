\part{Kernel}

\chapter{Overview}

操作系统的内核的目的在于管理硬件与提供系统核心功能,用戶可以根椐系統规划编译合适的内核。

\begin{lstlisting}[language=bash]
# make mrproper
# make clean
\end{lstlisting}


Linux内核其实是一个文件,并且在开机时直接被读入内存中,因此用户可以在获取内核源代码后编译自己的内核,同时也增加了系统稳定性和执行效率。

内核是操作系统中除BIOS之外最早被加载到内存中的,其中包含了所有可以让硬件与软件运作的信息,硬件的驱动程序可以编译为内核模块,也可以直接编译入内核。


\section{Directory Structure}


Linux内核源代码的目录结构如下:

\begin{compactitem}
\item arch:与硬件平台有关的选项
\item block:与块设备相关的设置数据
\item crypto:内核支持的加密技术(例如md5或des等)
\item Documentation:帮助文档
\item drivers:硬件的驱动程序
\item firmware:固件数据
\item fs:内核支持的文件系统(例如vfat、nfs、reiserfs等)
\item include:头(header)文件
\item init:内核初始化的定义功能(例如挂载与init程序的调用等)
\item ipc:Linux操作系统程序的通信
\item kernel:内核的程序、状态、线程、调度、信号等
\item lib:函数库
\item mm:与内存单元有关的数据(包括swap和虚拟内存等)
\item net:与网络有关的协议的数据,以及防火墙模块(net/ipv4/netfilter/*)
\item security:包含selinux在内的安全性设置
\item sound:与音效相关的模块
\item virt:与虚拟化相关的数据
\end{compactitem}

