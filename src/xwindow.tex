\part{X Window}


\chapter{Overview}

X Window System(简称为X或X11),并且本身也是客户端/服务器架构,因此可以划分为X Server和X Client两部分。

对于Linux来说,X Window System仅仅是一个软件,不过其本身具有的网络功能使其具有跨网络和跨平台的特性。

X Window System最初由MIT于1984年发布并且到1987年时进化到X11,后续的改进都是基于X11进行的,新版的X11R6于1994年发布。

XFree86的目标是“X+Free Software+x86 Hardware”,并且提供了早期的Linux使用的X Window System的主要内核,在Xorg基金会接手X11R6的维护后发布了X11R6.8版本及X11R7.x版本。

\begin{compactitem}
\item X11本身是软件而非操作系统;
\item X11通过网络来进行绘制和生成图形界面。
\end{compactitem}

事实上,X Server启动端口6000来与X Client交互,不过本机运行的X Window System将端口改为套接字(socket)。

如果操作系统上安装了多个X Client(例如Gnome和KDE等),就会使用6001和6002等端口来处理交互。

\begin{table}[htbp]
\centering
\begin{tabular}{|l|l|l|l|}
\hline
X Window System & Interface & Terminal & Port \\
\hline
第一个X & hostnome:0 & tty7 & 6000\\
\hline
第二个X & hostname:1 & tty8 & 6001\\
\hline
\end{tabular}
\end{table}



\section{Component}


基本上,X Window System可以划分为X Server和X Client。

\begin{compactitem}
\item X Server屏幕绘制、提供字体和管理硬件(例如鼠标、键盘等)。
\item X Client发送应用程序界面信息给X Server,X Server输出结果给显示硬件来绘制图形界面。
\end{compactitem}

X Server管理客户端硬件时将接收到的来自键盘/鼠标的输入信息转换为图形绘制到屏幕上,但是X Server本身并不知道客户端硬件对显示的影响,只是将硬件设备的行为发送到X Client并让其处理X Server的“事件”。

X Client处理来自X Server的操作并将操作结果转换为绘图数据后回传给X Server,因此X Client也可以称为X Application。

不同的X Client之间互不影响,而且X Client与X Server的隔离使其不需要了解硬件配置与操作系统,只是单纯地处理绘图的数据,本身并不绘图。

除了使用/etc/sysconfig目录下的keyboard/mouse等设置文件来管理硬件之外,X Window也提供了自己的设置文件。


Window Manager也是X Client,只是它负责全部X Client的管理,以及提供其他特殊功能。

\begin{compactitem}
\item 控制显示元素(例如任务栏、桌面背景等);
\item 管理虚拟桌面(virtual desktop);
\item 提供窗口控制参数(例如窗口大小、背景显示、窗口移动和窗口的最小(大)化等。
\end{compactitem}

不同的Window Manager使用的显示引擎不尽相同,因此X Client、X Server和Window Manager以及图形应用程序等构成了X Window System。

在桌面系统中,Display Manager提供的登录的环境并加载用户选择的Window Manager和语系等数据,现在Gnome和KDE等都提供了自己的Display Manager(例如Gnome Display Manager(gdm)等)。


\section{Configure}

默认情况下,X Server的设置文件位于/etc/X11目录下,相关的显示模块则主要位于/usr/lib/xorg/modules/(或/usr/lib64/xorg/modules/)目录。

\begin{compactitem}
\item 屏幕字体:/usr/share/X11/fonts/
\item 显卡芯片组:/usr/lib/xorg/modules/drivers/
\end{compactitem}

CentOS提供了chkfontpath命令来获取当前系统的字体文件目录,并且使用统一的设置文件(/etc/X11/xorg.conf)来规范X Server。

\section{startx}

在启动X Window System时\footnote{在CentOS和RedHat等系统时,需要启动xfs服务。},必须首先启动管理硬件和绘图的X Server,然后才能加载X Client。

在命令行界面调用startx可以启动X,startx可以主动查找用户或系统默认的X Server和X Client的设置文件来初始化X,而且用户也可以手动指定startx的外接参数。

\begin{lstlisting}[language=bash]
startx [ [ client ] options ... ] [ -- [ server ] [ display ] [ -listen ] options ... ]
\end{lstlisting}

startx查找的设置值可用顺序在X Server和X Client方面是不同的。

\begin{compactitem}
\item X Server

\begin{compactenum}
\item startx后接参数
\item $\sim$/.xserverrc
\item /etc/X11/xinit/xserverrc
\item /usr/bin/X
\end{compactenum}

\item X Client

\begin{compactitem}
\item startx后接参数
\item $\sim$/.xinitrc
\item /etc/X11/xinit/xinitrc
\item xterm
\end{compactitem}

\end{compactitem}



\section{xinit}

实际上,在startx查找和提供了设置文件后,系统会通过调用xinit来启动X。


\begin{lstlisting}[language=bash]
xinit [ [ client ] options ... ] [ -- [ server ] [ display ] options ... ]
\end{lstlisting}


在默认情况下,用户主文件夹下没有.xinitrc等设置文件,因此执行startx实际上调用的命令如下:


\begin{lstlisting}[language=bash]
# xinit /etc/X11/xinit/xinitrc -- /etc/X11/xinit/xserverrc
\end{lstlisting}

如果xserverrc设置文件也不存在,那么实际上调用的命令如下:

\begin{lstlisting}[language=bash]
# xinit /etc/X11/xinit/xinitrc -- /usr/bin/X
\end{lstlisting}

相比xinit,startx的方便之处在于它可以自动快速地找到相应的设置参数,如果单纯执行xinit产生的命令如下:

\begin{lstlisting}[language=bash]
# xinit xterm -geometry +1+1 -n login -display :0 -- x :0
\end{lstlisting}
其中,xterm是X下的虚拟终端机,后面的参数说明xterm的位置和是否登录等信息,最后的\texttt{-display~:0}表示xterm启动在“第:0号的X显示窗口”。

/usr/bin/X是Xorg的链接文件,Xorg是X Server的主程序,xinit命令的效果是启动X Server和加载X Client。


X Window System最先需要启动的是X Server,X Server的启动脚本通过/etc/X11/xinit/下的xserverrc来获得的,然后才能加载X Client到X Server。

用户执行/usr/bin/X就是执行系统最原始的X Server可执行文件,Xorg读取/etc/X11/xorg.conf。

默认情况下,用户主文件夹中不存在.xinitrc,因此在启动X Client时将以/etc/X11/xinit/xinitrc作为启动X Client的默认脚本。

xinitrc加载其他的文件参数(例如/etc/X11/xinit/xinitrc-common、/etc/X11/xinit/Xclients和/etc/sysconfig/desktop等),用户可以通过修改/etc/sysconfig/desktop中的DESKTOP参数来设置默认使用的Window Manager。

通常情况下,X Client就是Gnome或KDE,X Server在没有安装Gnome和KDE时就会使用twm窗口管理器来管理用户环境。







































































