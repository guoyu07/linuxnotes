\part{X Window}


\chapter{Overview}

X Window System(简称为X或X11),并且本身也是客户端/服务器架构,因此可以划分为X Server和X Client两部分。

对于Linux来说,X Window System仅仅是一个软件,不过其本身具有的网络功能使其具有跨网络和跨平台的特性。

X Window System最初由MIT于1984年发布并且到1987年时进化到X11,后续的改进都是基于X11进行的,新版的X11R6于1994年发布。

XFree86的目标是“X+Free Software+x86 Hardware”,并且提供了早期的Linux使用的X Window System的主要内核,在Xorg基金会接手X11R6的维护后发布了X11R6.8版本及X11R7.x版本。

\begin{compactitem}
\item X11本身是软件而非操作系统;
\item X11通过网络来进行绘制和生成图形界面。
\end{compactitem}

事实上,X Server启动端口6000来与X Client交互,不过本机运行的X Window System将端口改为套接字(socket)。

如果操作系统上安装了多个X Client(例如Gnome和KDE等),就会使用6001和6002等端口来处理交互。

\begin{table}[htbp]
\centering
\begin{tabular}{|l|l|l|l|}
\hline
X Window System & Interface & Terminal & Port \\
\hline
第一个X & hostnome:0 & tty7 & 6000\\
\hline
第二个X & hostname:1 & tty8 & 6001\\
\hline
\end{tabular}
\end{table}



\section{Component}


基本上,X Window System可以划分为X Server和X Client。

\begin{compactitem}
\item X Server负责屏幕绘制、提供字体和管理硬件(例如鼠标、键盘等)。
\item X Client发送应用程序信息给X Server,X Server输出结果给显示硬件来绘制图形界面。
\end{compactitem}

X Server管理客户端硬件时将接收到的来自键盘/鼠标的输入信息转换为图形绘制到屏幕上,但是X Server本身并不知道客户端硬件对显示的影响,只是将硬件设备的行为发送到X Client并让其处理X Server的“事件”。

X Client处理来自X Server的操作并将操作结果转换为绘图数据后回传给X Server,因此X Client也可以称为X Application。

X Client之间互相独立,而且不同的X Client之间互不影响,而且X Client与X Server的隔离使其不需要了解硬件配置与操作系统,只是单纯地处理绘图的数据,本身并不绘图。

除了使用/etc/sysconfig目录下的keyboard/mouse等设置文件来管理硬件之外,X Window也提供了自己的设置文件。

用户可以启动多个X,每个X显示位置(例如:0、:1等)使用-display来指定


Window Manager也是X Client,只是它负责全部X Client的管理,以及提供其他特殊功能(例如窗口的堆叠、移动和最大(小)化等)。

\begin{compactitem}
\item 控制显示元素(例如任务栏、桌面背景等);
\item 管理虚拟桌面(virtual desktop);
\item 提供窗口控制参数(例如窗口大小、背景显示、窗口移动和窗口的最小(大)化等。
\end{compactitem}

不同的Window Manager使用的显示引擎不尽相同,因此X Client、X Server和Window Manager以及图形应用程序等构成了X Window System。

在桌面系统中,Display Manager提供的登录的环境并加载用户选择的Window Manager和语系等数据,现在Gnome和KDE等都提供了自己的Display Manager(例如Gnome Display Manager(gdm)等)。


\section{Configure}

默认情况下,X Server的设置文件位于/etc/X11目录下,相关的显示模块则主要位于/usr/lib/xorg/modules/(或/usr/lib64/xorg/modules/)目录。

\begin{compactitem}
\item 屏幕字体:/usr/share/X11/fonts/
\item 显卡芯片组:/usr/lib/xorg/modules/drivers/
\end{compactitem}

CentOS提供了chkfontpath命令来获取当前系统的字体文件目录,并且使用统一的设置文件(/etc/X11/xorg.conf)来规范X Server。

\begin{lstlisting}[language=bash]
# X -version
X.Org X Server 1.16.3
Release Date: 2014-12-20
X Protocol Version 11, Revision 0
Build Operating System:  3.17.8-300.bz1178975.fc21.x86_64 
Current Operating System: Linux theqiong 3.18.7-200.fc21.x86_64 #1 SMP Wed Feb 11 21:53:17 UTC 2015 x86_64
Kernel command line: BOOT_IMAGE=/vmlinuz-3.18.7-200.fc21.x86_64 root=/dev/mapper/fedora_theqiong-root ro rd.lvm.lv=fedora_theqiong/swap rd.lvm.lv=fedora_theqiong/root rhgb quiet LANG=en_US.UTF-8
Build Date: 31 January 2015  11:23:27PM
Build ID: xorg-x11-server 1.16.3-2.fc21 
Current version of pixman: 0.32.6
	Before reporting problems, check http://wiki.x.org
	to make sure that you have the latest version.
\end{lstlisting}

/etc/X11/xorg.conf的内容按段落进行划分,每个段落以Section开始,并以EndSection结束,中间则包含相关设置值。

\begin{lstlisting}[language=bash]
Section "section name''
...
EndSection
\end{lstlisting}

\begin{table}[htbp]
\centering
\caption{X Server的常见设置}
\begin{tabular}{|l|l|}
\hline
Module & 加载到X Server中的模块(某些功能的驱动程序)\\
\hline
InputDevice& 键/鼠的格式、以及其他相关输入设备\\
\hline
Files & 设置字体目录等\\
\hline
Monitor & 监视器的格式(例如水平和垂直更新频率等),与硬件有关\\
\hline
Device & 显卡芯片组的相关设置\\
\hline
Screen & 屏幕的相关分辨率与色彩深度等,与显示的行为有关\\
\hline
ServerLayout & 当前X Server使用的选项\\
\hline
\end{tabular}
\end{table}


\begin{lstlisting}[language=bash]
Section "ServerLayout"
	Identifier     "X.org Configured"
	Screen      0  "Screen0" 0 0
	InputDevice    "Mouse0" "CorePointer"
	InputDevice    "Keyboard0" "CoreKeyboard"
EndSection

Section "Files"
	ModulePath   "/usr/lib64/xorg/modules"
	FontPath     "catalogue:/etc/X11/fontpath.d"
	FontPath     "built-ins"
EndSection

Section "Module"
	Load  "glx"
EndSection

Section "InputDevice"
	Identifier  "Keyboard0"
	Driver      "kbd"
EndSection

Section "InputDevice"
	Identifier  "Mouse0"
	Driver      "mouse"
	Option	    "Protocol" "auto"
	Option	    "Device" "/dev/input/mice"
	Option	    "ZAxisMapping" "4 5 6 7"
EndSection

Section "Monitor"
	Identifier   "Monitor0"
	VendorName   "Monitor Vendor"
	ModelName    "Monitor Model"
EndSection

Section "Device"
        ### Available Driver options are:-
        ### Values: <i>: integer, <f>: float, <bool>: "True"/"False",
        ### <string>: "String", <freq>: "<f> Hz/kHz/MHz",
        ### <percent>: "<f>%"
        ### [arg]: arg optional
        #Option     "SWcursor"           	# [<bool>]
        #Option     "HWcursor"           	# [<bool>]
        #Option     "NoAccel"            	# [<bool>]
        #Option     "ShadowFB"           	# [<bool>]
        #Option     "VideoKey"           	# <i>
        #Option     "WrappedFB"          	# [<bool>]
        #Option     "GLXVBlank"          	# [<bool>]
        #Option     "ZaphodHeads"        	# <str>
        #Option     "PageFlip"           	# [<bool>]
        #Option     "SwapLimit"          	# <i>
        #Option     "AsyncUTSDFS"        	# [<bool>]
        #Option     "AccelMethod"        	# <str>
	Identifier  "Card0"
	Driver      "nouveau"
	BusID       "PCI:1:0:0"
EndSection

Section "Screen"
	Identifier "Screen0"
	Device     "Card0"
	Monitor    "Monitor0"
	SubSection "Display"
		Viewport   0 0
		Depth     1
	EndSubSection
	SubSection "Display"
		Viewport   0 0
		Depth     4
	EndSubSection
	SubSection "Display"
		Viewport   0 0
		Depth     8
	EndSubSection
	SubSection "Display"
		Viewport   0 0
		Depth     15
	EndSubSection
	SubSection "Display"
		Viewport   0 0
		Depth     16
	EndSubSection
	SubSection "Display"
		Viewport   0 0
		Depth     24
	EndSubSection
EndSection
\end{lstlisting}

在X Server中启动X Font Server(XFS)时,可以执行如下的命令来启动xfs服务,从而可以为X Server提供字体库。



\begin{lstlisting}[language=bash]
# /etc/init.d/xfs start
\end{lstlisting}

X Server使用的字体其实是XFS服务提供,因此X Server需要在启动XFS服务后才能顺利启动。

\begin{compactitem}
\item XFS的主设置文件在/etc/X11/fs/config
\item 字体文件保存在/usr/share/X11/fonts/中
\item XFS的启动脚本位于/etc/init.d/xfs中
\end{compactitem}

\begin{lstlisting}[language=bash]
# vim /etc/X11/fs/config
client-limit = 10
clone-self = on
catalogue = /usr/share/X11/fonts/misc:unscaled,
	/usr/share/X11/fonts/75dpi:unscaled,
	/usr/share/X11/fonts/100dpi:unscaled,
	/usr/share/X11/fonts/Type1,
	/usr/share/X11/fonts/TTF,
	/usr/share/fonts/default/Type1
default-point-size = 120
default-resolutions = 75,75,100,100
deferglyphs = 16
use-syslog = on
no-listen = tcp
\end{lstlisting}

用户可以使用chkfontpath列出当前支持的字体文件,并且可以追加新的字体文件目录,这样在使用fontconfig提供的font-cache来新建字体缓存文件后就可以使用新的字体。

\begin{lstlisting}[language=bash]
# chkfontpath
# mkfontdir
# font-cache -fsv
# chkfontpath -a /usr/share/fonts/macfonts
# /etc/init.d/xfs restart
# xlsofnts | grep 'Songti'
\end{lstlisting}

X Server本身可以主动以内置的模块来检测系统硬件,并将硬件与字体的检测结构写入/root/xorg.conf.new,而且可以测试新的文件是否可以顺利运行。

\begin{lstlisting}[language=bash]
# X -configure :1
# X -config /root/xorg.conf.new :1
\end{lstlisting}


\section{startx}

在启动X Window System时\footnote{在CentOS和RedHat等系统中,如果在xorg.conf中没有直接写入字体的路径,就需要启动xfs服务来使用字体服务器。},必须首先启动管理硬件和绘图的X Server,然后才能加载X Client。

在命令行界面调用startx可以启动X,startx可以主动查找用户或系统默认的X Server和X Client的设置文件来初始化X,而且用户也可以手动指定startx的外接参数。

\begin{lstlisting}[language=bash]
startx [ [ client ] options ... ] [ -- [ server ] [ display ] [ -listen ] options ... ]
\end{lstlisting}

startx查找的设置值可用顺序在X Server和X Client方面是不同的。

\begin{compactitem}
\item X Server

\begin{compactenum}
\item startx后接参数
\item $\sim$/.xserverrc
\item /etc/X11/xinit/xserverrc
\item /usr/bin/X
\end{compactenum}

\item X Client

\begin{compactitem}
\item startx后接参数
\item $\sim$/.xinitrc
\item /etc/X11/xinit/xinitrc
\item xterm
\end{compactitem}

\end{compactitem}

显示器的分辨率等可以使用gtf来进行调整。

\begin{lstlisting}[language=bash]
gtf h-resolution v-resolution refresh [-v|--verbose] [-f|--fbmode] [-x|--xorgmode]
\end{lstlisting}



\section{xinit}

实际上,在startx查找和提供了设置文件后,系统会通过调用xinit来启动X。


\begin{lstlisting}[language=bash]
xinit [ [ client ] options ... ] [ -- [ server ] [ display ] options ... ]
\end{lstlisting}


在默认情况下,用户主文件夹下没有.xinitrc等设置文件,因此执行startx实际上调用的命令如下:


\begin{lstlisting}[language=bash]
# xinit /etc/X11/xinit/xinitrc -- /etc/X11/xinit/xserverrc
\end{lstlisting}

如果xserverrc设置文件也不存在,那么实际上调用的命令如下:

\begin{lstlisting}[language=bash]
# xinit /etc/X11/xinit/xinitrc -- /usr/bin/X
\end{lstlisting}

相比xinit,startx的方便之处在于它可以自动快速地找到相应的设置参数,如果单纯执行xinit产生的命令如下:

\begin{lstlisting}[language=bash]
# xinit xterm -geometry +1+1 -n login -display :0 -- x :0
\end{lstlisting}
其中,xterm是X下的虚拟终端机,后面的参数说明xterm的位置和是否登录等信息,最后的\texttt{-display~:0}表示xterm启动在“第:0号的X显示窗口”。

/usr/bin/X是Xorg的链接文件,Xorg是X Server的主程序,xinit命令的效果是启动X Server和加载X Client。


X Window System最先需要启动的是X Server,X Server的启动脚本通过/etc/X11/xinit/下的xserverrc来获得的,然后才能加载X Client到X Server。

用户执行/usr/bin/X就是执行系统最原始的X Server可执行文件,Xorg读取/etc/X11/xorg.conf。

默认情况下,用户主文件夹中不存在.xinitrc,因此在启动X Client时将以/etc/X11/xinit/xinitrc作为启动X Client的默认脚本。

xinitrc加载其他的文件参数(例如/etc/X11/xinit/xinitrc-common、/etc/X11/xinit/Xclients和/etc/sysconfig/desktop等),用户可以通过修改/etc/sysconfig/desktop中的DESKTOP参数来设置默认使用的Window Manager。

通常情况下,X Client就是Gnome或KDE,X Server在没有安装Gnome和KDE时就会使用twm窗口管理器来管理用户环境。



\chapter{Driver}

在计算机中,硬件驱动程序与内核有关,因此需要使用Development Tools配合kernel-devel等来编译驱动程序。

\section{Nvidia}


Xorg基金会针对Nvidia显卡驱动程序提供了nv模块,但是无法提供其他功能,因此用户需要额外安装Nvidia提供的显卡驱动程序。

在安装显卡驱动程序时,用户可以选择是否允许Nvidia编译内核模块,或者自己来编译内核模块,而且还需要安装额外的OpenGL函数库来支持OpenGL应用。

在编译过程中,Nvidia显卡驱动程序主动修改xorg.conf中的Device段落中的Driver项,并且使用“nvidia”来替换“nv”,最终在/usr/lib64/xorg/modules/drivers/目录下新增nvidia\_drv.so。

在升级驱动程序时,Nvidia显卡驱动程序自带的nvidia-installer可以帮助用户更新驱动。



\begin{lstlisting}[language=bash]
# nvidia-installer --update
\end{lstlisting}


\section{AMD}

AMD在收购ATI显卡后以Open Source形式开放了显卡驱动程序,用户可以自行编译安装。




\section{Intel}

Intel显卡驱动程序使用Open Source形式开放源代码,并且一般以libdrm命名驱动程序文件。


\begin{lstlisting}[language=bash]
# locate libdrm
...
/usr/lib/libdrm.so.2
/usr/lib/libdrm.so.2.4.0
/usr/lib/libdrm_intel.so.1
/usr/lib/libdrm_intel.so.1.0.0
/usr/lib/libdrm_nouveau.so.2
/usr/lib/libdrm_nouveau.so.2.0.0
/usr/lib/libdrm_radeon.so.1
/usr/lib/libdrm_radeon.so.1.0.1
/usr/lib64/libdrm.so
/usr/lib64/libdrm.so.2
/usr/lib64/libdrm.so.2.4.0
/usr/lib64/libdrm_intel.so
/usr/lib64/libdrm_intel.so.1
/usr/lib64/libdrm_intel.so.1.0.0
/usr/lib64/libdrm_nouveau.so
/usr/lib64/libdrm_nouveau.so.2
/usr/lib64/libdrm_nouveau.so.2.0.0
/usr/lib64/libdrm_radeon.so
/usr/lib64/libdrm_radeon.so.1
/usr/lib64/libdrm_radeon.so.1.0.1
...
\end{lstlisting}




























































